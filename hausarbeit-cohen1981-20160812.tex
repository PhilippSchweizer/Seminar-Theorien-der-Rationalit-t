\documentclass[ngerman,12pt, titlepage, smallheadings, nomath]{scrartcl}
\usepackage[]{libertine}
\usepackage{amssymb,amsmath}
\usepackage{ifxetex,ifluatex}
\usepackage{fixltx2e} % provides \textsubscript
\ifnum 0\ifxetex 1\fi\ifluatex 1\fi=0 % if pdftex
  \usepackage[T1]{fontenc}
  \usepackage[utf8]{inputenc}
\else % if luatex or xelatex
  \usepackage{fontspec}
  \defaultfontfeatures{Ligatures=TeX,Scale=MatchLowercase}
\fi
% use upquote if available, for straight quotes in verbatim environments
\IfFileExists{upquote.sty}{\usepackage{upquote}}{}
% use microtype if available
\IfFileExists{microtype.sty}{%
\usepackage{microtype}
\UseMicrotypeSet[protrusion]{basicmath} % disable protrusion for tt fonts
}{}
\usepackage{hyperref}
\hypersetup{unicode=true,
            pdftitle={Kompetenz und Performanz im Schlussfolgern},
            pdfauthor={Philipp Schweizer},
            pdfborder={0 0 0},
            breaklinks=true}
\urlstyle{same}  % don't use monospace font for urls
\ifnum 0\ifxetex 1\fi\ifluatex 1\fi=0 % if pdftex
  \usepackage[shorthands=off,main=ngerman]{babel}
\else
  \usepackage{polyglossia}
  \setmainlanguage[]{german}
\fi
\usepackage{biblatex}

\addbibresource{rationality.bib}
\IfFileExists{parskip.sty}{%
\usepackage{parskip}
}{% else
\setlength{\parindent}{0pt}
\setlength{\parskip}{6pt plus 2pt minus 1pt}
}
\setlength{\emergencystretch}{3em}  % prevent overfull lines
\providecommand{\tightlist}{%
  \setlength{\itemsep}{0pt}\setlength{\parskip}{0pt}}
\setcounter{secnumdepth}{5}
% Redefines (sub)paragraphs to behave more like sections
\ifx\paragraph\undefined\else
\let\oldparagraph\paragraph
\renewcommand{\paragraph}[1]{\oldparagraph{#1}\mbox{}}
\fi
\ifx\subparagraph\undefined\else
\let\oldsubparagraph\subparagraph
\renewcommand{\subparagraph}[1]{\oldsubparagraph{#1}\mbox{}}
\fi
\usepackage[german=guillemets]{csquotes}
\usepackage[width=14cm, height=23cm]{geometry}
\usepackage[doublespacing]{setspace}

\title{Kompetenz und Performanz im Schlussfolgern}
\providecommand{\subtitle}[1]{}
\subtitle{Zur Konsequenz einer Unterscheidung von L. Jonathan Cohen}
\author{Philipp Schweizer}
\publishers{Essay im Seminar ``Theorien der Rationalität'' von Prof.~Dr.~Thomas
Sturm, SoSe 2016, Goethe-Universität Frankfurt/M., Institut für
Philosophie}
\date{}

\begin{document}
\maketitle

{
\setcounter{tocdepth}{3}
\tableofcontents
}
\newpage

\section{Einleitung}\label{einleitung}

\vspace{-1.25em}

Eine Nachbarin hat sich aus ihrer Wohnung ausgesperrt. Der Ofen steht
auf 300 °C und das Backgut droht zu verbrennen. In ihrer Verzweiflung
läuft sie auf die Straße, um Hilfe zu holen und sieht, dass ich auf dem
Balkon sitze. Ganz aufgelöst und außer Atem kommt sie auf mich zu und
bittet mich um Hilfe. Über den Balkon im Erdgeschoss erreiche ich ihre
Wohnung im ersten Stock und kann die Tür von innen öffnen (die Balkontür
ist geöffnet).

Hat meine Nachbarin vernünftig gehandelt? Konnte sie sich darauf
verlassen, dass ich kletternd den ersten Stock erreichen würde? Anhand
welcher Kriterien, auf Grundlage welcher Normen können wir ihr Handeln
beurteilen? Und als Psychologen: sollten wir ihr Handeln, und das diesem
zugrunde liegende Schlussfolgern, überhaupt beurteilen? Die Fragen die
sich an diesen Fall richten lassen, verweisen auf das, wofür L. Jonathan
Cohen in seinem vieldiskutierten Artikel \emph{Can Human Irrationality
Be Experimentally Demonstrated?} argumentiert
\autocite*{cohen1981a}\footnote{Zwischen 1981 und 1987 veröffentliche
  die Zeitschrift \emph{The Behavioral and Brain Sciences} 44 offene
  Peer-Kommentare zu Cohens ursprünglichem Artikel, sowie vier Antworten
  von Cohen \autocites{cohen1981b}{cohen1983}{cohen1984}{cohen1987}.
  Seitenangaben zu Cohen ohne Jahr beziehen sich im Folgenden immer auf
  \autocite*{cohen1981a}.}. Darin richtet er sich gegen eine Strömung
innerhalb der kognitiven und experimentellen Psychologie, die die
Fehlerhaftigkeit menschlicher Rationalität betont, festgestellt in
Laborstudien in denen Probanden Denksportaufgaben lösen sollten. Die
Vorreiter dieser Strömung, Amos Tversky und Daniel Kahneman,
konstatierten in ihren Experimenten vermeintlich allgemein und manifest
auftretende Verzerrungen des deduktiven und probabilistischem
Schlussfolgern normaler Menschen und führten diese auf kognitive
Heuristiken zurück \autocite*{tversky1974}. So führe zum Beispiel die
Repräsentativitätsheuristik zu falschen Schlüssen im sogenannten
Linda-Problem in dem Probanden eine Personenbeschreibung von Linda
gegeben wird (die sie unter anderem als sozial engagiert beschreibt) und
sie dann einschätzen sollen, ob es wahrscheinlicher ist, dass Linda

\begin{enumerate}
\def\labelenumi{\alph{enumi})}
\item
  Bankangestellte, oder
\item
  Bankangestellte und feministische Aktivistin ist.
\end{enumerate}

\textcite{kahneman2012} argumentiert in diesem Fall, dass es mit
\emph{a)} eine im Sinne der Logik der Wahrscheinlichkeit richtige
Antwort gibt, weil die Menge feministischer Bankangestelter vollständig
in der Menge der Bankangestelten enthalten ist. Kahneman berichtet, dass
er und Tversky überrascht waren festzustellen, \enquote{dass 89 Prozent
der Studenten in {[}der{]} Stichprobe gegen die Logik der
Wahrscheinlichkeit verstießen.} \autocite*[Kap. 15]{kahneman2012}

Gegen die Ansicht, dass dieses und andere ähnlich gelagerte Experimente
beweisen, dass menschliches Schließen systematisch fehlerhaft ist,
wendet sich Cohen mit seinem Artikel. Er nennt es eine
erkenntnistheoretische Binsenweisheit, dass normative Kriterien der
Alltagsableitbarkeit und - wahrscheinlichkeit mit Intuitionen normaler
Menschen übereinstimmen müssen. Ausgehend davon kommt man nicht umhin,
normalen Erwachsenen eine grundsätzlich korrekte deduktive und
probabilsitische Kompetenz zuzuschreiben \autocite[362]{cohen1981b}. Zur
Stützung dieses Arguments unterscheidet Cohen zwischen Kompetenz und
Performanz des menschlichen logischen Denkens, eine Unterscheidung die
Chomsky in die Linguistik eingeführt hatte (S. 321;
\textcite{eells1991}, S. 11). Auf Grundlage einer normativen Theorie der
menschlichen rationalen Kompetenz, die Aussagen darüber trifft, wozu
normale Erwachsene kognitiv in der Lage sind, lässt sich niemals die
systematische Fehlerhaftigkeit eben dieser Kompetenz feststellen, weil
sie ihre Normativität, ihren Bewertungsmaßstab aus alltäglichen
Intuitionen bezieht: das normale menschliche Schließen setzt seine
eigenen Standards (S. 317).

Dieses \emph{a priori} Argument für menschliche Kompetenz in deduktivem
und probabilitischem Schließen ist Gegenstand dieses Essays
\autocite[vgl.][S. 68]{shier2000}.\footnote{\enquote{It is an a priori
  epistemological argument about the limitations on any community's
  judgments about certain of its own practices, and ultimately it says
  nothing about the actual character of human reasoning. Therefore, even
  if his argument were granted, Cohen should not be seen as defending
  human reasoning from its critics, but instead as rendering human
  reasoning totally indefensible.}} Dabei steht die Frage im
Mittelpunkt, ob Cohen berechtigt ist zu behaupten, dass psychologische
Experimente niemals die systematische Fehlerhaftigkeit menschlicher
Rationalität zeigen können (S. 330)\footnote{\enquote{However, nothing
  in the existing literature on cognitive reasoning, or in any possible
  future results of human experimental enquiry, could have bleak
  implications for human rationality, in the sence of implications that
  establish a faulty competence.}} und das experimentelle Psychologen
überhaupt nichts über diese Kompetenz aussagen können (S.
321).\footnote{\enquote{This factual theory of competence will be just
  as idealised as the normative theory from which it derives. And though
  it is a contribution to the psychology of cognition it is a by-product
  of the logical or philosophical analysis of norms rather than
  something that experimentally oriented psychologists need to devote
  effort to constructing. It is not only all the theory of competence
  that is needed in its area. It is also all that is possible, since a
  different competence, if it actually existed, would just generate
  evidence that called for a revision of the corresponding normative
  theory.}}

\section{Cohens Argument für rationale
Kompetenz}\label{cohens-argument-fuxfcr-rationale-kompetenz}

\vspace{-1.25em}

\subsection{Intuitionen als Grundlage}\label{intuitionen-als-grundlage}

\vspace{-1.25em}

\begin{enumerate}
\def\labelenumi{\arabic{enumi}.}
\item
  Welche Qualitäten müssen normative Theorien des logischen Denkens
  aufweisen?
\item
  Sie müssen \ldots{} Intuitionen
\item
  \begin{enumerate}
  \def\labelenumii{\alph{enumii}.}
  \setcounter{enumii}{3}
  \item
    \begin{enumerate}
    \def\labelenumiii{\alph{enumiii}.}
    \setcounter{enumiii}{7}
    \tightlist
    \item
      die beweisgrundlage einer solchen normativen theorie bilden
      wirkliche Fälle logischen Denkens wie man es bei \enquote{Laien}
      beobachten kann. Solche Fälle nennt Cohen \emph{Intuitionen}, also
      die \emph{spontane und ungeschulte Neigung} ohne Beweis oder
      Schlussfolgerung über einen logischen oder probabilitischen
      Sachverhalt \emph{zu urteilen} (S. 318).
    \end{enumerate}
  \end{enumerate}
\item
  diese Intuitionen werden im Zuge der theoriebildung nicht bewertet,
  sondern interpretiert: aus ihnen will man ableiten, wozu normale
  Menschen im logischen Denken in der Lage sind im Unterschied zu dem,
  wozu sie sich in der lage zeigen. aus diesen intuitionen soll die
  allgemeine kompetenz des logischen denkens abgeleitet werden.
\item
  jetzt wird auch klar, warum diese kompetenz nicht für grundsätlich
  fehlerhaft gehalten werden kann und auch nicht der fehlerhaftigkeit im
  experiment überführt werden kann: die Standards die uns zu ihrer
  bewertung zur verfügung stehen, sind von ihr selbst gesetzt (im
  beispiel john-mini-rolls: beides intuitionen die von der theorie
  erklärt werden müssen und in der normativen theorie des logischen
  denkens auch gar nicht im widerspruch stehen)
\item
  vor dem hintergrund dieser kompetenz, gültig zu schließen, können wir
  nun fehler, auch systematische fehler, damit erklären, dass die
  kompetenz, bzw. die theorie mit der wie sie beschreiben, natürlich
  idealisiert ist: eine fehlfunktion eines informationsverarbeitenden
  mechanismus muss angenommen und ihre erklärung gesucht werden (317).
\end{enumerate}

\begin{quote}
\enquote{The fact is that conditions are rarely, if ever, ideal for the
exercise of such a competence. {[}\ldots{}{]} a variety of factors may
interfere with the excercise of a competence for deductive or
probabilistic reasoning.} (S. 322)
\end{quote}

\begin{enumerate}
\def\labelenumi{\arabic{enumi}.}
\setcounter{enumi}{3}
\tightlist
\item
  weil die Intuitionen zur Grundlage eine notwendige Bedingung einer
  normativen theorie des logischen Denkens sind, ist eine
\end{enumerate}

In der Tat geht Cohen nicht darauf ein, \emph{wie} ein solcher Fall
beschaffen sein muss, um als Beweis dienen zu können: muss es sich um
weitverbreitete Ansichten handeln? Wieviel Interpretation steckt in
einer Intuition, bevor sie als Beweis dienen kann? Wie wird mit diesem
Problem umgegangen?

Die Thesen die Cohen aufstellt

\begin{enumerate}
\def\labelenumi{\arabic{enumi}.}
\item
  kein ergebniss experimenteller Untersuchung (des Menschen) kann eine
  fehlerhafte kompetenz implizieren (S. 330)
\item
  die faktische theorie der kompetenz ist zwar ein beitrag zur
  psychologie des erkenntnisvermögens (cognition), aber nichts worum
  sich experimentell orientierte Philosophen kümmern müssten. denn diese
  faktische theorie und die normative theorie von der sie stammt, von
  der sie abgeleitet wird, ist ein nebenprodukt der logischen oder
  philosophischen analyse von normen (S. 321).
\end{enumerate}

\textbf{\enquote{the credentials of those normative theories \ldots{}}}

\begin{enumerate}
\def\labelenumi{\arabic{enumi}.}
\setcounter{enumi}{2}
\item
  Eine normative Theorie, mit der die Rationalität oder Irrationalität
  von Laien und ihren naiven Schlussfolgerungen oder
  Wahrscheinlichkeitsurteilen bewertet wird, muss in entscheidenden
  Punkten mit Beweisen im Einklang sein, die aus Fällen ungeschulter
  Intuition gewonnen werden (S. 317).
\item
  Daraus folgt, dass normales menschliches logisches Denken (reasoning)
  nicht für fehlerhaft programmiert gehalten werden kann: es setzt seine
  eigenen Standards (S. 317).
\end{enumerate}

\subsection{Kompetenz und Performanz}\label{kompetenz-und-performanz}

\vspace{-1.25em}

\section{Einwände gegen Cohens
Argument}\label{einwuxe4nde-gegen-cohens-argument}

\vspace{-1.25em}

\subsection{Erster Einwand}\label{erster-einwand}

\vspace{-1.25em}

\subsection{Zweiter Einwand}\label{zweiter-einwand}

\vspace{-1.25em}

\section{Fazit}\label{fazit}

\vspace{-1.25em}

\newpage

\section*{Bibliographie}\label{bibliographie}
\addcontentsline{toc}{section}{Bibliographie}

\vspace{-1.25em}

\printbibliography

\end{document}
